\documentclass{thesisclass}
% Based on thesisclass.cls of Timo Rohrberg, 2009
% ----------------------------------------------------------------
% Thesis - Main document
% ----------------------------------------------------------------

\usepackage{pgfplots}
\usepackage{listings}
\usetikzlibrary{trees,arrows,positioning, calc, shapes.geometric}

\usepackage{ntheorem}
\pgfdeclarelayer{background}
\pgfsetlayers{background,main}
\pgfplotsset{width=7cm}

%\theoremstyle{definition}
\newtheorem{definition}{Definition}
\newtheorem{definition-non}{Definition}
%\theoremstyle{plain}
\newtheorem{theorem}{Theorem}
\newtheorem{theorem-non}{Theorem}
\newtheorem{lemma}{Lemma}

\definecolor{lightred}{rgb}{1,0.7,0.7}
\definecolor{lightblue}{rgb}{0.7,0.7,1.0}
\definecolor{lightgreen}{rgb}{0.7,1,0.7}

\newcommand{\convexpath}[2]{
[   
    create hullnodes/.code={
        \global\edef\namelist{#1}
        \foreach [count=\counter] \nodename in \namelist {
            \global\edef\numberofnodes{\counter}
            \node at (\nodename) [draw=none,name=hullnode\counter] {};
        }
        \node at (hullnode\numberofnodes) [name=hullnode0,draw=none] {};
        \pgfmathtruncatemacro\lastnumber{\numberofnodes+1}
        \node at (hullnode1) [name=hullnode\lastnumber,draw=none] {};
    },
    create hullnodes
]
($(hullnode1)!#2!-90:(hullnode0)$)
\foreach [
    evaluate=\currentnode as \previousnode using \currentnode-1,
    evaluate=\currentnode as \nextnode using \currentnode+1
    ] \currentnode in {1,...,\numberofnodes} {
-- ($(hullnode\currentnode)!#2!-90:(hullnode\previousnode)$)
  let \p1 = ($(hullnode\currentnode)!#2!-90:(hullnode\previousnode) - (hullnode\currentnode)$),
    \n1 = {atan2(\x1,\y1)},
    \p2 = ($(hullnode\currentnode)!#2!90:(hullnode\nextnode) - (hullnode\currentnode)$),
    \n2 = {atan2(\x2,\y2)},
    \n{delta} = {-Mod(\n1-\n2,360)}
  in 
    {arc [start angle=\n1, delta angle=\n{delta}, radius=#2]}
}
-- cycle
}

% Node Styles
\tikzstyle{root}=[draw,fill=gray,rectangle,minimum size=8pt,inner sep=0pt]
\tikzstyle{read}=[draw,fill=blue,circle,minimum size=8pt,inner sep=0pt]
\tikzstyle{mod}=[draw,fill=magenta,circle,minimum size=8pt,inner sep=0pt]
\tikzstyle{memo}=[draw,fill=green!50,shape=diamond,minimum size=8pt,inner sep=0pt]
\tikzstyle{write}=[draw,fill=orange,circle,minimum size=8pt,inner sep=0pt]
\tikzstyle{par}=[draw,fill=yellow,shape=diamond,minimum size=8pt,inner sep=0pt]

\usepackage{makecell}

%% -------------------------------
%% |  Information for PDF file   |
%% -------------------------------
\hypersetup{
 pdfauthor={Emanuel J\"obstl},
 pdftitle={A Practical Approach to Analyzing Incremental Programs Using Execution Traces},
 pdfsubject={Not Set},
 pdfkeywords={Not Set}
}

%% ---------------------------------
%% |  about the thesis  |
%% ---------------------------------

\newcommand{\myname}{Emanuel J\"obstl}
\newcommand{\mytitle}{A Practical Approach to Analyzing Incremental Programs Using Execution Traces}
\newcommand{\myinstitute}{Institute for Program Structures\\
													and Data Organization (IPD)}

\newcommand{\reviewerone}{Prof. Dr. Walter F. Tichy}
\newcommand{\reviewertwo}{Prof. Dr. Ralf H. Reussner}
\newcommand{\advisor}{Prof. Dr. Walter F. Tichy}
\newcommand{\advisortwo}{Umut A. Acar}

\newcommand{\timestart}{24. Juli 2014}
\newcommand{\timeend}{23. November 2014}
\newcommand{\submissiontime}{23. November. 2014}


%% ---------------------------------
%% | ToDo Marker - only for draft! |
%% ---------------------------------
% Remove this section for final version!
\setlength{\marginparwidth}{20mm}

\newcommand{\margtodo}
{\marginpar{\textbf{\textcolor{red}{ToDo}}}{}}

\newcommand{\todo}[1]
{{\textbf{\textcolor{red}{(\margtodo{}#1)}}}{}}


%% --------------------------------
%% | Old Marker - only for draft! |
%% --------------------------------
% Remove this section for final version!
\newenvironment{deprecated}
{\begin{color}{gray}}
{\end{color}}


%% --------------------------------
%% | Settings for word separation |
%% --------------------------------
% Help for separation:
% In german package the following hints are additionally available:
% "- = Additional separation
% "| = Suppress ligation and possible separation (e.g. Schaf"|fell)
% "~ = Hyphenation without separation (e.g. bergauf und "~ab)
% "= = Hyphenation with separation before and after
% "" = Separation without a hyphenation (e.g. und/""oder)

% Describe separation hints here:
\hyphenation{
% Pro-to-koll-in-stan-zen
% Ma-na-ge-ment  Netz-werk-ele-men-ten
% Netz-werk Netz-werk-re-ser-vie-rung
% Netz-werk-adap-ter Fein-ju-stier-ung
% Da-ten-strom-spe-zi-fi-ka-tion Pa-ket-rumpf
% Kon-troll-in-stanz
}


%% ------------------------
%% |    Including files   |
%% ------------------------
% Only files listed here will be included!
% Userful command for partially translating the document (for bug-fixing e.g.)
\includeonly{%
titlepage,
declaration,
introduction,
content,
appendix
}


%%%%%%%%%%%%%%%%%%%%%%%%%%%%%%%%%
%% Here, main documents begins %%
%%%%%%%%%%%%%%%%%%%%%%%%%%%%%%%%%
\begin{document}

% Remove the following line for German text
\selectlanguage{english}

\frontmatter
\pagenumbering{roman}
\include{titlepage}
\include{declaration}

\chapter*{Acknowledgement}

Umut A. Acar was a very helpful supervisor and mentioned the idea of experimenting with different properties inside node tags during a discussion, adding substential value to this writing. Thank you!

Thomas Marshal provided a lot of explanation on the concepts, inner workings and implementation details of TBD. Thank you!

Prof. Dr. Walter F. Tichy provided very valuable feedback regarding the style and readability of this writing. Thank you!

Johannes Rudolph from the Scala community and Eugene Burmako from the Scala development team helped provided a lot of information that helped me to understand and apply Scala macros. Thank you!

The Baden-W\"urttemberg Stiftung and the interACT program made it possible for me to write my bachelor's thesis at Carnegie Mellon University. Thank you!

\begin{figure}
\centering
\includegraphics[scale=2]{logos/BW.jpg}
\end{figure}


\chapter*{Abstract}
The principle of incremental programming has been investigated for many years. A recent approach is the concept of \textit{self-adjusting programs}. Self-adjusting programs are programs that automatically adjust their output and internal state if the input changes. In this writing, we inspect \textit{TBD}, a platform that provides an API to write self-adjusting programs. We show how we can use dependency graphs of TBD program runs to calculate \textit{trace distance}, a metric for changes between two executions, in practice. Furthermore, we define multiple types of trace distance that also allow us to analyze and optimize programs. Finally, we evaluate the usefulness of our implementation of trace distance using basic algorithms. 

\vspace{10 mm}

Das Prinzip der inkrementellen Programmierung ist bereits seit Jahrzehnten bekannt. Ein neuartiger Ansatz ist das Konzept der \textit{Self-Adjusting Programs}. Self-Adjusting Programs sind Programme, die automatisch ihre Ausgabe sowie ihren internen Status an Eingabe\"anderungen anpassen. In dieser Arbeit betrachten wir \textit{TBD}, eine Plattform, die eine Schnittstelle zur Programmierung von Self-Adjusting Programs bereitstellt. Wir zeigen zudem, wie wir Abh\"angigkeitsgraphen von TBD Programausf\"uhrungen benutzen k\"onnen, um \textit{Trace Distance}, eine Metrik f\"ur die verschiedenheit zweier Programmausf\"uhrungen, zu in der Praxis berechnen. Zudem werden mehrere Arten von Trace Distance definiert, die es uns erlauben das Programm zu analysieren und zu optimieren. Am Ende der Arbeit wird die N\"utzlichkeit sowie Korrektheit unserer Implementierung von Trace Distance anhand einiger einfacher Algorithmen evaluiert. 


%% -------------------
%% |   Directories   |
%% -------------------
\tableofcontents
\blankpage


%% -----------------
%% |   Main part   |
%% -----------------
\mainmatter
\pagenumbering{arabic}

\chapter{Introduction}
\label{ch:Introduction}
%[Explain Target: Incremental Computation, Change Propagation and Trace Distance] 

Trace distance is an important metric for incremental programs. This metric measures the re-execution cost that occours when the input to an incremental program changes and the program is partially re-executed. The objective of this work is to provide a software tool capable of calculating trace distance for certain incremental programs.  

The term \textit{incremental program} refers to a program that after a change of the program input only re-evaluates the computation steps that depend on the changed parts of the input. Therefore, such a program can adjust the output faster than it would be possible by re-executing the whole program with the new input. In particular, it can is possible to adjust to constant-sized updates in constant time, even if a re-execution would run in polynomial time complexity or worse. The process of adjusting the program output and program state to input changes is called \textit{change propagation}. Incremental programs are also called \textit{dynamic programs} in literature. 

\textit{Trace distance} can be roughly described as metric for the amount of differences in the internal program state between two executions of the same program with differing inputs. Since the changes in program state reflect the re-evaluated computation steps, this number is directly related to the time needed for change propagation. Trace distance can therefore be used as benchmark for incremental programs. 

\section{A simple Incremental Program}
\label{sec:simple_example}
Consider a simple program that takes a list, consisting of nodes $a_0$ to $a_n$ as input. Each list node has a numeric value and a successor. The program reads each node, and applies a mapping function $g$ to the value of the node, and outputs a list containing the results. The functionality of this program is also implemented as \textit{map operator} in many programming languages. For this example, let $g$ be a simple function, only multiplying the value by two. 

Executing this program for a list always has a time complexity of $O(n)$, even if only a single element $a_i$ changes. An incremental version of this program would, however, find the single call $f_i$ that is affected by the update of node $a_i$, re-execute this single call, and update the output accordingly. If an element is inserted into the input, $g$ will also be re-executed, and the new element will be inserted. If an element is removed from the input, the corresponding element will simply be removed from the output. With this approach updating the output has time complexity $O(k)$, where $k$ is the size of the change in the input. This implies that we can update the output in constant time, as long as the changes to the input have constant size. 

Note that this program is particularly easy to incrementalize, since there are no dependencies between different calls to our mapping function $g$. 
\section{Approaches to Incremental Computation}
%Since incremental programs have been subject to research for over 30 years, different approaches exist. 

%[Introduce Terminology: Self-Adjusting Computation]

A well-known form of creating incremental programs is to design an incremental algorithm for a single purpose. Such algorithms are carefully modified to hold enough internal state to make change propagation possible. A simple example for such an algorithm would be a basic insertion-sort algorithm. For insertion-sort, a newly added input value can simply be inserted into the output, a value removed from the input can be removed from the output and an updated input value can be removed and then added again. More sophisticated examples would be so-called \textit{kinetic algorithms} or \textit{kinetic data structures}. Those data structures are for instance used to perform geometric calculations, such as finding bounding boxes or intersection areas, on groups of moving objects, as described in \cite{yu2008practical} and \cite{basch2004kinetic} respectively.

\subsection{Self-Adjusting Computation}

However, in \cite{Acar2005thesis}, Acar pointed out that the utilization of such algorithms can be cumbersome. The development of incremental algorithms might be complex, the algorithms are usually single purpose and they cannot be composed to a bigger applications in general. Therefore, the development of languages or platforms that automate the process of deriving incremental programs from non-incremental programs is of interest. After such a transformation, the program should be capable of adjusting its internal state and the program output to reflect changes of the program input. This particular approach to incremental computation is called \textit{self-adjusting computation}.

For accomplishing self-adjusting computation, tracking control and data dependencies through the program and storing intermediate results can be used. These techniques are called \textit{dynamic dependence graphs (DDGs)} and \textit{memoization}, respectively. 

In 2014, an self-adjusting computation platform called \textit{TBD} was developed at Carnegie Mellon University. TBD is an abbreviation for \textit{To Be Determined}. The platform is implemented in the \textit{Scala} language and makes use of DDGs and memoization to accomplish self-adjusting computation. 

\section{Contents of this Work}

This work focuses on the development of a tool that is capable of calculating trace distance for programs implemented on top of TBD. First, the principles of self-adjusting computation will be explained in depth, especially the process change propagation using DDGs and memoization. Second, the program interface of the TBD platform will be described. Third, our approach of calculating trace-distance will be discussed. Then, we discuss our implementation including the framework for data collection using Scala macros. 
Finally, we will show how our tool can be used for analyzing algorithms by discussing the implementation of a $map$, a $reduce$ and a $sort$ function.  
%
\chapter{Self-Adjusting Programs}
\label{ch:self_adjusting}

The principles of self-adjusting programs using DDGs and memoization was described by Acar in \cite{Acar2005thesis}. This chapter provides a summary of his work. 
When changing the input set of a self-adjusting program, the program has to adjust the internal state and the program output to match the new input. This requires that the internal state of the program has to be known before propagating input changes. 

Since functions which depend on certain input variables have to be located, control and data dependencies have to be known, too. Therefore, the program execution has to be tracked carefully, including functions which are being called. It is not straight-forward to detect and control reads and writes to abritary memory locations. Therefore, the need for a custom program model which enables us to track variables and functions arises.

\section{Closure Machine Model}

In a classical RAM model, a program can write and read abritary memory locations, any number of times. This makes it very hard to track dependencies. Even if tracking of dependencies is accomplished for a RAM model, the program can still overwrite memory locations which hold crucial information for program state, for example the results of previous computations. Therefore, the model for self-adjusting computation uses labeled memory locations, which can only be written once. 

Functions can usually be referenced by their memory location, however this does not include the referencing environment of the function, like paramteres or free variables. The model solves this shortcoming by providing a set of functions to manipulate variables using a function or closure which is passed as parameter. Therefore the model is called the \textit{Closure Machine}. 

\section{Directed Dependence Graphs}

\section{Memoization}

\section{Change Propagation}

\section{Example}

\chapter{The TBD Platform}
\label{ch:tbd_platform}

\section{Program interface}

\chapter{Calculating Trace Distance}
\label{ch:implementation}

\section{Node Tags}

\section{Trace Distance}

\chapter{Implementation}
\label{ch:impl}

\section{Architecture}

\section{Data Collection}

\subsection{Usage of Scala Macros}

\chapter{Use Cases}
\label{ch:use_cases}

\chapter{Discussion}
\label{ch:discussion}


%% --------------------
%% |   Bibliography   |
%% --------------------
\cleardoublepage
\phantomsection
\addcontentsline{toc}{chapter}{\bibname}

\iflanguage{english}
{\bibliographystyle{IEEEtranSA}}	% english style
{\bibliographystyle{babalpha-fl}}	% german style
												  
% Use IEEEtran for numeric references
%\bibliographystyle{IEEEtranSA})

\bibliography{thesis}


%% ----------------
%% |   Appendix   |
%% ----------------
\cleardoublepage


\appendix

\addchap{Appendix}	% english style


\section{Proofs of Correctness}
\label{app:proofs}
		
%\setcounter{figure}{0}





\end{document}
