
\appendix

\addchap{Appendix}	% english style

\section{Proofs of Correctness}
\label{sec:correctness}
\subsection{Correctness of Pure Trace Distance} 

For two DDGs $A$ and $B$, let $\delta_p'(A, B)$ be the count of nodes that were re-executed or deleted by TBD when transforming DDG $A$ into DDG $B$. We show that $\delta_p'(A, B)$ $=$ $\delta_p(A, B)$. 

\begin{lemma}[$\delta_p'(A, B)$ $\le$ $\delta_p(A, B)$]
Proof: Each node re-executed by TBD is assigned a new ID. At least, all nodes with a new ID are counted. Therefore, $\delta_p'(A, B)$ $\le$ $\delta_p(A, B)$ holds.
\end{lemma}

\begin{lemma}[$\delta_p'(A, B)$ $\ge$ $\delta_p(A, B)$]
Proof: The trace distance algorithm greedily removes all nodes with equal ids from both DDGs and counts all nodes with new Id's at most. Only re-executed nodes get new IDs. Therefore, $\delta_p'(A, B)$ $\ge$ $\delta_p(A, B)$.
\end{lemma} 

\begin{theorem}[$\delta_p'(A, B)$ $=$ $\delta_p(A, B)$]
Proof: From $\delta_p'(A, B)$ $\ge$ $\delta_p(A, B)$ and $\delta_p'(A, B)$ $\le$ $\delta_p(A, B)$ we conclude $\delta_p'(A, B)$ $=$ $\delta_p(A, B)$ 
\end{theorem}

\subsection{Correctness of Memo-Sensitive Trace Distance} 

\begin{definition}[Change propagation algorithm]
Let $\Sigma$ be a change propagation algorithm. A change propagation algorithm may only re-order\footnote{in TBD, re-ordering happens when a memo-match occurs and a node is attached to another subtree}, re-execute or delete nodes from an existing trace.
\end{definition}

\begin{definition}[Memoizing change propagation algorithm]
Let $\Sigma_m$ be a memoizing change propagation algorithm. This means that $\Sigma_m$ memoizes all nodes and only re-evaluates nodes that have to be re-evaluated and automatically holds change propagation where necessary. 
\end{definition}

For two DDGs $A$ and $B$, let $\delta_m'(A, B)$ be the count of nodes that were re-executed or deleted by a memoizing change propagation algorithm $\Sigma_m$ when transforming DDG $A$ into DDG $B$. We show that $\delta_m'(A, B)$ $=$ $\delta_p(A, B)$. 

\begin{lemma}[Two equal trace nodes execute equally and make equal sub-calls.]
\label{lem:node_determinism}
Proof: If the nodes are equal, they refer to same TBD function, read, write, mod, memo or par. These functions are guaranteed to execute the same way if the input parameters are equal. Also, the tag of the node contains all parameters the function depends on, including free variables and higher order functions. Nodes are only equal if the tag equals. Accessing static or class variables from inside function parameters can be ruled out due to the constraints listed in chapter \ref{ch:tbd_platform}.
\end{lemma}

\begin{lemma}[$\delta_m'(A, B)$ $\ge$ $\delta_m(A, B)$]
Proof: Let $Y$ be the set off all nodes of $A$ without an equal node in $B$. Let $R$ be the set of all nodes of $B$ without an equal node in $A$. That means, that for all nodes in $Y$ there is no corresponding node with an equal tag in DDG $B$ and vice versa. Therefore, we have to at least re-evaluate all nodes within $Y$ and $R$, where the removal of a node counts as re-evaluation. 
\end{lemma}


\begin{lemma}[$\delta_m'(A, B)$ $\le$ $\delta_m(A, B)$]
Proof: Let $T$ be the set of all nodes in $A$ and that have an equal node in $B$ and vice versa. Lets assume that the is a node $v$ in $T$ which is re-executed during change propagation by the memoizing optimal algorithm $\Sigma*$. Due to lemma \ref{lem:node_determinism} we know that this re-execution was unnecessary. Therefore, our assumption was wrong and we know now that an memoizing algorithm $\Sigma*$ does not re-evaluate nodes in $A$ and that have an equal node in $B$ and vice versa. From this it follows that only nodes that do not exist in both DDGs have to be re-evaluated. Since the count of all nodes that do not exist in both DDGs equals $\delta_m(A, B)$, we know that the count of re-evaluated nodes, or $\delta_m'(A, B)$ is lower or equal $\delta_m(A, B)$. 
\end{lemma}

\begin{theorem}[$\delta_m'(A, B)$ $=$ $\delta_m(A, B)$]
Proof: From $\delta_m'(A, B)$ $\ge$ $\delta_m(A, B)$ and $\delta_m'(A, B)$ $\le$ $\delta_m(A, B)$ we conclude $\delta_m'(A, B)$ $=$ $\delta_m(A, B)$ 
\end{theorem}

\subsection{Correctness of Allocation-Sensitive Trace Distance}

\begin{definition}[Allocating change propagation algorithm]
Let $\Sigma_a$ be an allocating change propagation algorithm. This means that $\Sigma_a$ chooses memory locations for modifiables in a way that there are no further invocations of change propagation due to the memory location. Also, $\Sigma_a$ memoizes all nodes like $\Sigma_m$. 
\end{definition}

\begin{lemma}[$\delta_a(A, B)$ $\le$ $\delta_m(A, B)$]
Proof: Since an allocating change propagating algorithm also memoizes each node line a memoizing change propagation algorithm, those nodes are also included in the matching. Since they are included in the matching, they are not counted by both algorithms. Therefore, allocation sensitive trace distance is lower or equal memoized trace distance for all pairs of DDGs $A$ and $B$. This also implies that the nodes counted by allocation sensitive trace distance are a subset of the nodes counted by memo-sensitive trace distance. 
\end{lemma}

For two DDGs $A$ and $B$, let $\delta_a'(A, B)$ be the count of nodes that were re-executed or deleted by an allocating change propagation algorithm $\Sigma_m$ when transforming DDG $A$ into DDG $B$. We show that $\delta_a'(A, B)$ $=$ $\delta_a(A, B)$. 

\begin{lemma}[$\delta_a'(A, B)$ $\le$ $\delta_a(A, B)$]
Proof: Assume there is a pair of nodes $(a, b)$ from $A$ and $B$ respectively, that is matched by allocation-sensitive trace distance but got re-executed. Since allocation-sensitive trace distance removes all references to modifiables from the node tags, and the tags of $a$ and $b$ were otherwise equal, the re-execution must have been induced by a difference in the location of a modifiable. This leads to a conflict with our assumption. Therefore, there exists no pair of nodes $(a, b)$ that is matched by allocation sensitive trace distance but getting re-executed, leading to the conclusion that the count of re-executed nodes is always lower or equal than allocation-sensitive trace distance. 
\end{lemma}

\begin{lemma}[$\delta_a'(A, B)$ $\ge$ $\delta_a(A, B)$]
Proof: Assume there is a pair of nodes $(a, b)$ from $A$ and $B$ respectively, that is not matched by allocation-sensitive trace distance but did not got re-executed. This means that the node tag is different and therefore, the parameters the node depend on have changed. A change propagation algorithm has to update the node, even when memoized. Therefore, the count of re-executed nodes is always greater or equal than allocation-sensitive trace distance. 
\end{lemma}

\begin{theorem}[$\delta_a'(A, B)$ $=$ $\delta_a(A, B)$]
Proof: From $\delta_a'(A, B)$ $\ge$ $\delta_a(A, B)$ and $\delta_a'(A, B)$ $\le$ $\delta_a(A, B)$ we conclude $\delta_a'(A, B)$ $=$ $\delta_a(A, B)$ 
\end{theorem}

\section{Evaluation Data}
\label{app:tests}

\subsection{Used Commands}
The data for the evaluation was generated using the commands from listing \ref{code:eval_commands} using the discussed analyzer framework. 

\begin{figure}
\begin{lstlisting}[frame=single,basicstyle=\tiny]
#Unmemoized map with allocation sensitive trace distance. 
bin/visualization.sh -i 50 -r -a snmap -o chart2d -t exhaustive 
	-d alloc > analysisOutput/MapAlloc.txt
#Unmemoized map with pure trace distance. 
bin/visualization.sh -i 50 -r -a snmap -o chart2d -t exhaustive 
	-d pure > analysisOutput/MapPure.txt
#Memoized map with allocation sensitive trace distance. 
bin/visualization.sh -i 50 -r -a smmap -o chart2d -t exhaustive 
	-d alloc > analysisOutput/MemoMapAlloc.txt
#Memoized map with pure trace distance. 
bin/visualization.sh -i 50 -r -a smmap -o chart2d -t exhaustive 
	-d pure > analysisOutput/MemoMapPure.txt
#Unmemoized linear reduce with allocation sensitive trace distance. 
bin/visualization.sh -i 50 -r -a snlreduce -o chart2d -t exhaustive 
	-d alloc > analysisOutput/LinearReduceAlloc.txt
#Unmemoized linear reduce with pure trace distance. 
bin/visualization.sh -i 50 -r -a snlreduce -o chart2d -t exhaustive 
	-d pure > analysisOutput/LinearReducePure.txt
#Memoized linear reduce with allocation sensitive trace distance. 
bin/visualization.sh -i 50 -r -a mlreduce -o chart2d -t exhaustive 
	-d alloc > analysisOutput/MemoLinearReduceAlloc.txt
#Memoized linear reduce with pure trace distance. 
bin/visualization.sh -i 50 -r -a mlreduce -o chart2d -t exhaustive 
	-d pure > analysisOutput/MemoLinearReducePure.txt
#Unmemoized tree reduce with allocation sensitive trace distance. 
bin/visualization.sh -i 50 -r -a sntreduce -o chart2d -t exhaustive 
	-d alloc > analysisOutput/TreeReduceAlloc.txt
#Unmemoized tree reduce with pure trace distance. 
bin/visualization.sh -i 50 -r -a sntreduce -o chart2d -t exhaustive 
	-d pure > analysisOutput/TreeReducePure.txt
#Memoized tree reduce with allocation sensitive trace distance. 
bin/visualization.sh -i 50 -r -a mtreduce -o chart2d -t exhaustive 
	-d alloc > analysisOutput/MemoTreeReduceAlloc.txt
#Memoized tree reduce with pure trace distance. 
bin/visualization.sh -i 50 -r -a mtreduce -o chart2d -t exhaustive 
	-d pure > analysisOutput/MemoTreeReducePure.txt
#Unmemoized quicksort with allocation sensitive trace distance. 
bin/visualization.sh -i 50 -r -a nqsort -o chart2d -t exhaustive 
	-d alloc > analysisOutput/QuicksortAlloc.txt
#Unmemoized quicksort with pure trace distance. 
bin/visualization.sh -i 50 -r -a nqsort -o chart2d -t exhaustive 
	-d pure > analysisOutput/QuicksortPure.txt
#Memoized quicksort with allocation sensitive trace distance. 
bin/visualization.sh -i 50 -r -a mqsort -o chart2d -t exhaustive 
	-d alloc > analysisOutput/MemoQuicksortAlloc.txt
#Memoized quicksort with pure trace distance. 
bin/visualization.sh -i 50 -r -a mqsort -o chart2d -t exhaustive 
	-d pure > analysisOutput/MemobQuicksortPure.txt
\end{lstlisting}
\caption{Commands used to generate the data used for evaluation}
\label{code:eval_commands}
\end{figure}



