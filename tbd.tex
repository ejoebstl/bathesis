\documentclass[conference]{IEEEtran}

\usepackage{hyperref}


  
% *** MISC UTILITY PACKAGES ***
%
%\usepackage{ifpdf}
% Heiko Oberdiek's ifpdf.sty is very useful if you need conditional
% compilation based on whether the output is pdf or dvi.
% usage:
% \ifpdf
%   % pdf code
% \else
%   % dvi code
% \fi
% The latest version of ifpdf.sty can be obtained from:
% http://www.ctan.org/tex-archive/macros/latex/contrib/oberdiek/
% Also, note that IEEEtran.cls V1.7 and later provides a builtin
% \ifCLASSINFOpdf conditional that works the same way.
% When switching from latex to pdflatex and vice-versa, the compiler may
% have to be run twice to clear warning/error messages.

% *** GRAPHICS RELATED PACKAGES ***
%
\ifCLASSINFOpdf
  \usepackage[pdftex]{graphicx}
  % declare the path(s) where your graphic files are
  %\graphicspath{{../img/}{../jpeg/}}
  % and their extensions so you won't have to specify these with
  % every instance of \includegraphics
  % \DeclareGraphicsExtensions{.pdf,.jpeg,.png}
\else
  % or other class option (dvipsone, dvipdf, if not using dvips). graphicx
  % will default to the driver specified in the system graphics.cfg if no
  % driver is specified.
  % \usepackage[dvips]{graphicx}
  % declare the path(s) where your graphic files are
  % \graphicspath{{../eps/}}
  % and their extensions so you won't have to specify these with
  % every instance of \includegraphics
  % \DeclareGraphicsExtensions{.eps}
\fi
% graphicx was written by David Carlisle and Sebastian Rahtz. It is
% required if you want graphics, photos, etc. graphicx.sty is already
% installed on most LaTeX systems. The latest version and documentation can
% be obtained at: 
% http://www.ctan.org/tex-archive/macros/latex/required/graphics/
% Another good source of documentation is "Using Imported Graphics in
% LaTeX2e" by Keith Reckdahl which can be found as epslatex.ps or
% epslatex.pdf at: http://www.ctan.org/tex-archive/info/
%
% latex, and pdflatex in dvi mode, support graphics in encapsulated
% postscript (.eps) format. pdflatex in pdf mode supports graphics
% in .pdf, .jpeg, .png and .mps (metapost) formats. Users should ensure
% that all non-photo figures use a vector format (.eps, .pdf, .mps) and
% not a bitmapped formats (.jpeg, .png). IEEE frowns on bitmapped formats
% which can result in "jaggedy"/blurry rendering of lines and letters as
% well as large increases in file sizes.
%
% You can find documentation about the pdfTeX application at:
% http://www.tug.org/applications/pdftex

% *** MATH PACKAGES ***
%
%\usepackage[cmex10]{amsmath}
% A popular package from the American Mathematical Society that provides
% many useful and powerful commands for dealing with mathematics. If using
% it, be sure to load this package with the cmex10 option to ensure that
% only type 1 fonts will utilized at all point sizes. Without this option,
% it is possible that some math symbols, particularly those within
% footnotes, will be rendered in bitmap form which will result in a
% document that can not be IEEE Xplore compliant!
%
% Also, note that the amsmath package sets \interdisplaylinepenalty to 10000
% thus preventing page breaks from occurring within multiline equations. Use:
%\interdisplaylinepenalty=2500
% after loading amsmath to restore such page breaks as IEEEtran.cls normally
% does. amsmath.sty is already installed on most LaTeX systems. The latest
% version and documentation can be obtained at:
% http://www.ctan.org/tex-archive/macros/latex/required/amslatex/math/

% *** SPECIALIZED LIST PACKAGES ***
%
%\usepackage{algorithmic}
% algorithmic.sty was written by Peter Williams and Rogerio Brito.
% This package provides an algorithmic environment fo describing algorithms.
% You can use the algorithmic environment in-text or within a figure
% environment to provide for a floating algorithm. Do NOT use the algorithm
% floating environment provided by algorithm.sty (by the same authors) or
% algorithm2e.sty (by Christophe Fiorio) as IEEE does not use dedicated
% algorithm float types and packages that provide these will not provide
% correct IEEE style captions. The latest version and documentation of
% algorithmic.sty can be obtained at:
% http://www.ctan.org/tex-archive/macros/latex/contrib/algorithms/
% There is also a support site at:
% http://algorithms.berlios.de/index.html
% Also of interest may be the (relatively newer and more customizable)
% algorithmicx.sty package by Szasz Janos:
% http://www.ctan.org/tex-archive/macros/latex/contrib/algorithmicx/

% *** ALIGNMENT PACKAGES ***
%
%\usepackage{array}
% Frank Mittelbach's and David Carlisle's array.sty patches and improves
% the standard LaTeX2e array and tabular environments to provide better
% appearance and additional user controls. As the default LaTeX2e table
% generation code is lacking to the point of almost being broken with
% respect to the quality of the end results, all users are strongly
% advised to use an enhanced (at the very least that provided by array.sty)
% set of table tools. array.sty is already installed on most systems. The
% latest version and documentation can be obtained at:
% http://www.ctan.org/tex-archive/macros/latex/required/tools/

%\usepackage{mdwmath}
%\usepackage{mdwtab}
% Also highly recommended is Mark Wooding's extremely powerful MDW tools,
% especially mdwmath.sty and mdwtab.sty which are used to format equations
% and tables, respectively. The MDWtools set is already installed on most
% LaTeX systems. The lastest version and documentation is available at:
% http://www.ctan.org/tex-archive/macros/latex/contrib/mdwtools/

% IEEEtran contains the IEEEeqnarray family of commands that can be used to
% generate multiline equations as well as matrices, tables, etc., of high
% quality.

%\usepackage{eqparbox}
% Also of notable interest is Scott Pakin's eqparbox package for creating
% (automatically sized) equal width boxes - aka "natural width parboxes".
% Available at:
% http://www.ctan.org/tex-archive/macros/latex/contrib/eqparbox/

% *** SUBFIGURE PACKAGE ***
% subfig.sty, also written by Steven Douglas Cochran, is the modern
% replacement for subfigure.sty. However, subfig.sty requires and
% automatically loads Axel Sommerfeldt's caption.sty which will override
% IEEEtran.cls handling of captions and this will result in nonIEEE style
% figure/table captions. To prevent this problem, be sure and preload
% caption.sty with its "caption=false" package option. This is will preserve
% IEEEtran.cls handing of captions. Version 1.3 (2005/06/28) and later 
% (recommended due to many improvements over 1.2) of subfig.sty supports
% the caption=false option directly:
\usepackage[caption=false,font=footnotesize]{subfig}
      
%
% The latest version and documentation can be obtained at:
% http://www.ctan.org/tex-archive/macros/latex/contrib/subfig/
% The latest version and documentation of caption.sty can be obtained at:
% http://www.ctan.org/tex-archive/macros/latex/contrib/caption/

% *** FLOAT PACKAGES ***
%
%\usepackage{fixltx2e}
% fixltx2e, the successor to the earlier fix2col.sty, was written by
% Frank Mittelbach and David Carlisle. This package corrects a few problems
% in the LaTeX2e kernel, the most notable of which is that in current
% LaTeX2e releases, the ordering of single and double column floats is not
% guaranteed to be preserved. Thus, an unpatched LaTeX2e can allow a
% single column figure to be placed prior to an earlier double column
% figure. The latest version and documentation can be found at:
% http://www.ctan.org/tex-archive/macros/latex/base/

%\usepackage{stfloats}
% stfloats.sty was written by Sigitas Tolusis. This package gives LaTeX2e
% the ability to do double column floats at the bottom of the page as well
% as the top. (e.g., "\begin{figure*}[!b]" is not normally possible in
% LaTeX2e). It also provides a command:
%\fnbelowfloat
% to enable the placement of footnotes below bottom floats (the standard
% LaTeX2e kernel puts them above bottom floats). This is an invasive package
% which rewrites many portions of the LaTeX2e float routines. It may not work
% with other packages that modify the LaTeX2e float routines. The latest
% version and documentation can be obtained at:
% http://www.ctan.org/tex-archive/macros/latex/contrib/sttools/
% Documentation is contained in the stfloats.sty comments as well as in the
% presfull.pdf file. Do not use the stfloats baselinefloat ability as IEEE
% does not allow \baselineskip to stretch. Authors submitting work to the
% IEEE should note that IEEE rarely uses double column equations and
% that authors should try to avoid such use. Do not be tempted to use the
% cuted.sty or midfloat.sty packages (also by Sigitas Tolusis) as IEEE does
% not format its papers in such ways.

% correct bad hyphenation here
\hyphenation{}

%\usepackage{multicol}
%\usepackage{multirow}
\usepackage{ntheorem}

%\usepackage{fancyvrb}
%\usepackage{longtable, lscape}
%\usepackage{setspace}
%\usepackage{float}

% Umkreiste Zahlen
\usepackage{tikz}
\newcommand*\circled[1]{\tikz[baseline=(char.base)]{
  \node[shape=circle,draw,inner sep=2pt] (char) {#1};}}

% Definition
\usepackage{amsmath} 
%\theoremstyle{definition}
   \newtheorem{definition}{Definition}
   \newtheorem{definition-non}{Definition}
%\theoremstyle{plain}
    \newtheorem{theorem}{Theorem}
    \newtheorem{theorem-non}{Theorem}  

% Euro-Zeichen
\usepackage{eurosym}

% Mathematische Symbole
\usepackage{amssymb}


\begin{document}

\title{A comparison of the incremental computation platforms Naiad and TBD}
%\\ by using surplus IT-resources}

\author{\IEEEauthorblockN{Walter Tichy, Umut Acar, Emanuel J\"{o}bstl}
\IEEEauthorblockA{Karlsruhe Institute of Technology, Faculty of Computer Science\\
Carnegie Mellon University, Computer Science Department\\
}}

% use for special paper notices
%\IEEEspecialpapernotice{(Invited Paper)}

% make the title area
\maketitle

\begin{abstract}
The principle of incremental programming has been well known for many years. While the automatic transformation of sequential programs into efficient incremental programs is not solved in general, a number of platforms for incremental computation have been created. In this document, we inspect the incremental computation platform TBD. First, we summarize known principles like directed dependency graphs, execution traces, intrinsic trace distance and trace stability, then we create an program model which enables us to apply those principles to TBD programs. We also present an algorithm to calculate the intrinsic trace distance in practice. Finally, we show how we can make automatic optimizations to programs by analyzing the dependency graph. 
\end{abstract}

\begin{IEEEkeywords}
Keywords. 
\end{IEEEkeywords}

\IEEEpeerreviewmaketitle
%
% -----------------------------
%

\section{Introduction}
This section should describe the purpose of incremental computation and also explain for which purpose the concept of incremental computation is useful. 

\subsection{Approaches to incremental computation}
This subsection should describe various historic approaches to incremental computation. This includes, but is not limited to: 

\subsection{Naiad and TBD}
This subsection should shortly outline Naiad and TBD. Additionally, the choice of comparing Naiad to TBD should be explained. 

\section{NAIAD}
This section should outline the Naiad framework and it's purpose. 

\subsection{Approach}
This subsection should shortly explain the approach Naiad uses, especially in regard of incremental computation. 
In other words, parallel and distributed features of Naiad should not be described in depth, because they are not a matter of interest for this theseis. 

\subsection{Programming model}
This subsection should describe the programming model of Naiad. The description should be from the view of a developer who creates software atop of the platform. The concepts and used programming patterns should be explained clearly, also an example should be provided. 

Frameworks build on top of Naiad, for example the Liandi framework should be mentioned. 
\section{TBD}
This section should outline the TBD framework, a framework for incremental computation currently being developed at CMU. The framework is being developed in the Scala language. This enables us to exploit the relection capabilties of Scala for analysis \cite{burmako2013scala}. 

\subsection{Programming interface}
This subsection should describe the TBD core API. Since TBD relies on the use of patterns known from functional languages, these patterns should be described. Also, the constraints for and the responsibilites of the developer have to be specified. The section should be concluded with an example. 

\subsection{Program model}
This subsection should describe a program model for TBD, which can be used for theoretical conclusions in the following sections. The differences and similarities with the programming model in section \ref{sec:ddg_memo} should be highlighted.
\section{Conducted experiments}
This section should outline important properties of algorithms used for this benchmark, and why these features are important. 

Note: The testcases listed here are subject to change. 

Each subsection should describe the differences and challenges of the implementations for Naiad and TBD. 

\subsection{Wordcount}
Wordcount was choosen as a benchmark because many machine-learning approaches rely on wordcount on big data. Furthermore, wordcount serves as a prime example of an algorithm implemented on top of map-reduce. 

\subsection{Sorting numbers}
A sorting algorithm (quicksort or mergesort, has yet to be decided) was chosen as benchmark because it introduces a special form of difficulties for incremental computation: If the input of a sorting operation changes, the dependency graph is expected to change.

\subsection{Pagerank}
Pagerank is an algorithm highly relevant for real-world applications. Futhermore, it is also prone to cascading updates, which makes it an interesting benchmark for incremental computation platforms. 
\section{Benchmark results}
This section should clearly describe the testing environment. Also, the input data to the tests should be characterized.

\subsection{Speed}
This section should describe the results in regard to processing time. If there are strong deviations during testing, the cause should be explained. 

\subsection{Memory}
This section should describe the results in regard to memory usage. This section should also expand on anomalies. 

\subsection{Code overhead}
Additionally to processing speed and memory usage, the difference of code overhead between the platform dependend implementations should be measured. 
\section{Conclusion}
This section should conclude with the findings of the executed benchmarks. Furthermore, it should provide a guideline for choosing one of the two frameworks, depending on which type of algorithm is to be implemented. 

\subsection{Future work}
The final section should shortly outline problems encountered but not solved during the writing of this theses, as well as encourage future research on interesting issues of incremental computation.

 


%\IEEEtriggeratref{8}
\bibliographystyle{IEEEtran}
\bibliography{tbd}
\end{document}