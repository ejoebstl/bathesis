\documentclass[conference]{IEEEtran}

\usepackage{hyperref}


  
% *** MISC UTILITY PACKAGES ***
%
%\usepackage{ifpdf}
% Heiko Oberdiek's ifpdf.sty is very useful if you need conditional
% compilation based on whether the output is pdf or dvi.
% usage:
% \ifpdf
%   % pdf code
% \else
%   % dvi code
% \fi
% The latest version of ifpdf.sty can be obtained from:
% http://www.ctan.org/tex-archive/macros/latex/contrib/oberdiek/
% Also, note that IEEEtran.cls V1.7 and later provides a builtin
% \ifCLASSINFOpdf conditional that works the same way.
% When switching from latex to pdflatex and vice-versa, the compiler may
% have to be run twice to clear warning/error messages.

% *** GRAPHICS RELATED PACKAGES ***
%
\ifCLASSINFOpdf
  \usepackage[pdftex]{graphicx}
  % declare the path(s) where your graphic files are
  %\graphicspath{{../img/}{../jpeg/}}
  % and their extensions so you won't have to specify these with
  % every instance of \includegraphics
  % \DeclareGraphicsExtensions{.pdf,.jpeg,.png}
\else
  % or other class option (dvipsone, dvipdf, if not using dvips). graphicx
  % will default to the driver specified in the system graphics.cfg if no
  % driver is specified.
  % \usepackage[dvips]{graphicx}
  % declare the path(s) where your graphic files are
  % \graphicspath{{../eps/}}
  % and their extensions so you won't have to specify these with
  % every instance of \includegraphics
  % \DeclareGraphicsExtensions{.eps}
\fi
% graphicx was written by David Carlisle and Sebastian Rahtz. It is
% required if you want graphics, photos, etc. graphicx.sty is already
% installed on most LaTeX systems. The latest version and documentation can
% be obtained at: 
% http://www.ctan.org/tex-archive/macros/latex/required/graphics/
% Another good source of documentation is "Using Imported Graphics in
% LaTeX2e" by Keith Reckdahl which can be found as epslatex.ps or
% epslatex.pdf at: http://www.ctan.org/tex-archive/info/
%
% latex, and pdflatex in dvi mode, support graphics in encapsulated
% postscript (.eps) format. pdflatex in pdf mode supports graphics
% in .pdf, .jpeg, .png and .mps (metapost) formats. Users should ensure
% that all non-photo figures use a vector format (.eps, .pdf, .mps) and
% not a bitmapped formats (.jpeg, .png). IEEE frowns on bitmapped formats
% which can result in "jaggedy"/blurry rendering of lines and letters as
% well as large increases in file sizes.
%
% You can find documentation about the pdfTeX application at:
% http://www.tug.org/applications/pdftex

% *** MATH PACKAGES ***
%
%\usepackage[cmex10]{amsmath}
% A popular package from the American Mathematical Society that provides
% many useful and powerful commands for dealing with mathematics. If using
% it, be sure to load this package with the cmex10 option to ensure that
% only type 1 fonts will utilized at all point sizes. Without this option,
% it is possible that some math symbols, particularly those within
% footnotes, will be rendered in bitmap form which will result in a
% document that can not be IEEE Xplore compliant!
%
% Also, note that the amsmath package sets \interdisplaylinepenalty to 10000
% thus preventing page breaks from occurring within multiline equations. Use:
%\interdisplaylinepenalty=2500
% after loading amsmath to restore such page breaks as IEEEtran.cls normally
% does. amsmath.sty is already installed on most LaTeX systems. The latest
% version and documentation can be obtained at:
% http://www.ctan.org/tex-archive/macros/latex/required/amslatex/math/

% *** SPECIALIZED LIST PACKAGES ***
%
%\usepackage{algorithmic}
% algorithmic.sty was written by Peter Williams and Rogerio Brito.
% This package provides an algorithmic environment fo describing algorithms.
% You can use the algorithmic environment in-text or within a figure
% environment to provide for a floating algorithm. Do NOT use the algorithm
% floating environment provided by algorithm.sty (by the same authors) or
% algorithm2e.sty (by Christophe Fiorio) as IEEE does not use dedicated
% algorithm float types and packages that provide these will not provide
% correct IEEE style captions. The latest version and documentation of
% algorithmic.sty can be obtained at:
% http://www.ctan.org/tex-archive/macros/latex/contrib/algorithms/
% There is also a support site at:
% http://algorithms.berlios.de/index.html
% Also of interest may be the (relatively newer and more customizable)
% algorithmicx.sty package by Szasz Janos:
% http://www.ctan.org/tex-archive/macros/latex/contrib/algorithmicx/

% *** ALIGNMENT PACKAGES ***
%
%\usepackage{array}
% Frank Mittelbach's and David Carlisle's array.sty patches and improves
% the standard LaTeX2e array and tabular environments to provide better
% appearance and additional user controls. As the default LaTeX2e table
% generation code is lacking to the point of almost being broken with
% respect to the quality of the end results, all users are strongly
% advised to use an enhanced (at the very least that provided by array.sty)
% set of table tools. array.sty is already installed on most systems. The
% latest version and documentation can be obtained at:
% http://www.ctan.org/tex-archive/macros/latex/required/tools/

%\usepackage{mdwmath}
%\usepackage{mdwtab}
% Also highly recommended is Mark Wooding's extremely powerful MDW tools,
% especially mdwmath.sty and mdwtab.sty which are used to format equations
% and tables, respectively. The MDWtools set is already installed on most
% LaTeX systems. The lastest version and documentation is available at:
% http://www.ctan.org/tex-archive/macros/latex/contrib/mdwtools/

% IEEEtran contains the IEEEeqnarray family of commands that can be used to
% generate multiline equations as well as matrices, tables, etc., of high
% quality.

%\usepackage{eqparbox}
% Also of notable interest is Scott Pakin's eqparbox package for creating
% (automatically sized) equal width boxes - aka "natural width parboxes".
% Available at:
% http://www.ctan.org/tex-archive/macros/latex/contrib/eqparbox/

% *** SUBFIGURE PACKAGE ***
% subfig.sty, also written by Steven Douglas Cochran, is the modern
% replacement for subfigure.sty. However, subfig.sty requires and
% automatically loads Axel Sommerfeldt's caption.sty which will override
% IEEEtran.cls handling of captions and this will result in nonIEEE style
% figure/table captions. To prevent this problem, be sure and preload
% caption.sty with its "caption=false" package option. This is will preserve
% IEEEtran.cls handing of captions. Version 1.3 (2005/06/28) and later 
% (recommended due to many improvements over 1.2) of subfig.sty supports
% the caption=false option directly:
\usepackage[caption=false,font=footnotesize]{subfig}
      
%
% The latest version and documentation can be obtained at:
% http://www.ctan.org/tex-archive/macros/latex/contrib/subfig/
% The latest version and documentation of caption.sty can be obtained at:
% http://www.ctan.org/tex-archive/macros/latex/contrib/caption/

% *** FLOAT PACKAGES ***
%
%\usepackage{fixltx2e}
% fixltx2e, the successor to the earlier fix2col.sty, was written by
% Frank Mittelbach and David Carlisle. This package corrects a few problems
% in the LaTeX2e kernel, the most notable of which is that in current
% LaTeX2e releases, the ordering of single and double column floats is not
% guaranteed to be preserved. Thus, an unpatched LaTeX2e can allow a
% single column figure to be placed prior to an earlier double column
% figure. The latest version and documentation can be found at:
% http://www.ctan.org/tex-archive/macros/latex/base/

%\usepackage{stfloats}
% stfloats.sty was written by Sigitas Tolusis. This package gives LaTeX2e
% the ability to do double column floats at the bottom of the page as well
% as the top. (e.g., "\begin{figure*}[!b]" is not normally possible in
% LaTeX2e). It also provides a command:
%\fnbelowfloat
% to enable the placement of footnotes below bottom floats (the standard
% LaTeX2e kernel puts them above bottom floats). This is an invasive package
% which rewrites many portions of the LaTeX2e float routines. It may not work
% with other packages that modify the LaTeX2e float routines. The latest
% version and documentation can be obtained at:
% http://www.ctan.org/tex-archive/macros/latex/contrib/sttools/
% Documentation is contained in the stfloats.sty comments as well as in the
% presfull.pdf file. Do not use the stfloats baselinefloat ability as IEEE
% does not allow \baselineskip to stretch. Authors submitting work to the
% IEEE should note that IEEE rarely uses double column equations and
% that authors should try to avoid such use. Do not be tempted to use the
% cuted.sty or midfloat.sty packages (also by Sigitas Tolusis) as IEEE does
% not format its papers in such ways.

% correct bad hyphenation here
\hyphenation{}

%\usepackage{multicol}
%\usepackage{multirow}
\usepackage{ntheorem}

%\usepackage{fancyvrb}
%\usepackage{longtable, lscape}
%\usepackage{setspace}
%\usepackage{float}

% Umkreiste Zahlen
\usepackage{tikz}
\newcommand*\circled[1]{\tikz[baseline=(char.base)]{
  \node[shape=circle,draw,inner sep=2pt] (char) {#1};}}

% Definition
\usepackage{amsmath} 
%\theoremstyle{definition}
   \newtheorem{definition}{Definition}
   \newtheorem{definition-non}{Definition}
%\theoremstyle{plain}
    \newtheorem{theorem}{Theorem}
    \newtheorem{theorem-non}{Theorem}  

% Euro-Zeichen
\usepackage{eurosym}

% Mathematische Symbole
\usepackage{amssymb}


\begin{document}

\title{Automated analysis of algorithms implemented on top of TBD}

\author{\IEEEauthorblockN{Walter Tichy, Umut Acar, Emanuel J\"{o}bstl}
\IEEEauthorblockA{Karlsruhe Institute of Technology, Faculty of Computer Science\\
Carnegie Mellon University, Computer Science Department\\
}}

\maketitle

\begin{abstract}
The principle of incremental programming has been well known for many years. While the automatic transformation of sequential programs into efficient incremental programs is not solved in general, a number of platforms for incremental computation have been created. In this document, we inspect the incremental computation platform TBD. First, we summarize known principles like directed dependency graphs, execution traces, intrinsic trace distance and trace stability, then we create an program model which enables us to apply those principles to TBD programs. We also present an algorithm to calculate the intrinsic trace distance in practice. Finally, we show how we can make automatic optimizations to programs by analyzing the dependency graph. 
\end{abstract}

\begin{IEEEkeywords}
Incremental computing, tbd, scala
\end{IEEEkeywords}

\IEEEpeerreviewmaketitle
%
% -----------------------------
%

\section{Introduction}
This section should describe the purpose of incremental computation and also explain for which purpose the concept of incremental computation is useful. 

\subsection{Approaches to incremental computation}
This subsection should describe various historic approaches to incremental computation. This includes, but is not limited to: 

\subsection{Naiad and TBD}
This subsection should shortly outline Naiad and TBD. Additionally, the choice of comparing Naiad to TBD should be explained. 

\section{Theoretic funcdamentals}
\label{sec:ddg_memo}
This section should give a brief summarization of the principles described in the work of U. Acar et al. This section is important, because we later have to adjust definitions found here to fit the program model of TBD. \cite{Acar2005thesis}

\subsection{Theoretical program model}
This subsection should describe the underlying theoretical program model. 

\subsection{Approach for incremental computation}
This subsection should describe the approaches of using Traces, DDGs (Directed Dependency Graphs) and memoization together to accomplish incremental computing. 

\subsection{Stable algorithms}
This subsection should describe the concepts of stable algorithms, intrinsic trace distance and their relationship. 

It should especially be made clear, that the intrinsic trace distance forms a lower bound for the time needed by change propagation during an update. 
\section{TBD}
This section should outline the TBD framework, a framework for incremental computation currently being developed at CMU. The framework is being developed in the Scala language. This enables us to exploit the relection capabilties of Scala for analysis \cite{burmako2013scala}. 

\subsection{Programming interface}
This subsection should describe the TBD core API. Since TBD relies on the use of patterns known from functional languages, these patterns should be described. Also, the constraints for and the responsibilites of the developer have to be specified. The section should be concluded with an example. 

\subsection{Program model}
This subsection should describe a program model for TBD, which can be used for theoretical conclusions in the following sections. The differences and similarities with the programming model in section \ref{sec:ddg_memo} should be highlighted.
\section{An intrinsic trace distance algorithm for TBD}
While \cite{Acar2005thesis} already outlines a greedy algorithm for calculating the intrinsic trace distance, there are details we have to take care of for accomplishing an implementation.

\subsection{Implementing node equality}
While equality of nodes is defined in section \ref{sec:node_equality}, it still remains open how a equality is implemented. For value types or objects we can use the $equals$ method provided by the Scala platform or define our own overload of $equals$, if needed.  

For testing anonymous functions passed to $read$, $memo$, $mod$, and $par$ for equality, we have to compare the the function itself, all parameters, all arguments, and all free variables bound from an outer scope, as described in definition \ref{def:fun_equality}. To accomplish this task, we can utilize the Scala macro API \cite{burmako2013scala}. Basically, the Scala macro API enables us to define small programs written in Scala, which are executed during compile time. From within these macros, we can access all information the compiler has and modify the abstract syntax tree (AST) of our program on the fly. 

To gather the necessary information for comparing anonymous functions during runtime, we replace the implementations of $read$, $memo$, $mod$ and $par$ with macros, which extract interesting information and create a tag from it. The macro generates code which calls the original function and passes the original parameters and the tag as arguments. 

During macro expansion, we can simply assign an unique ID to each function. This way, we can easily check whether to function tags refer to the same function. The arguments of the function are also well known for all methods provided by TBD, so they can be easily added to the tag. 

Finding free variables which are bound from an outer scope, however, is not straight-forward, because at the macro expansion step, the Scala compiler has no knowledge about whether a symbol is a function or a variable, or from where it is bound.

To extract only the correct symbols, we first create a list of all symbols which occur in the anonymous function $F$ and store them in a set $V = (v_1, ..., v_n)$. Then, for each $v_i$, $i \in[1..n]$, we iterate over all ancestors of $v_i$ in the AST of $F$. If we find a variable definition or parameter which defines a symbol with the same name as $v_i$, we know that $v_i$ is not bound from an outer scope, so we remove it from our list $V$.

Then, we iterate over all ancestors in the AST of the outermost enclosing scope of $F$. This scope is the class in which $F$ is defined in most cases. If we find an ancestor which defines a variable or parameter, we add the symbol of that variable to a set $D = (d_1, ..., d_m)$. Finally, we compute the set $U = (u_1, ..., u_k) = V \cap D$, whereas equality of elements in $V$ and $D$ is defined by equality of the symbol name. The set $U$ now contains only symbols, which are used in $F$, defined somewhere outside $F$ and are variables. 

Now, we generate code to add the name and value of each symbol $u_i$ to a Scala list, whereas the list is then added to the tag. The tag is passed to the original function, which adds it to the corresponding node in the DDG.  

By applying the described technique, we are now able to create a tag, which can be used to compare nodes which depend on anonymous functions for equality. 

\subsection{Implementing the intrinsic distance algorithm}
Given all nodes in two traces $T_1$ and $T_2$, including their tag, the trace distance can be computed like described in \cite{Acar2005thesis}.  

For a naive greedy algorithm, we create a tree- or hash set $S_1$, which holds all nodes from $T_1$. Then, we test for each node in $T_2$, if a node with an equal tag existed in $S_1$. If so, we remove the node from $S_1$.

When all nodes have been tested, the intrinsic trace distance is given by the size of the set $|S_1|$ plus the count of nodes from $T_2$ which were not contained in $S_1$. 
\subsection{Proof of correctness}
Whe have to show that our distance algorithm forms a lower bound for the count of nodes re-evaluated during change propagation. 

\begin{lemma}
\label{lem:equalExec}
Two equal nodes and execute equally and make equal subcalls
\end{lemma}

[Todo: Make this more formal]
Proof: We can safely assume this, because if the nodes are equal, they refer to same TBD mod, $read$, $write$, $mod$, $memo$ or $par$. These functions are guaranteed to execute the same way if the input parameters are equal. Also, the tag of the node contains all parameters on which the function depends, including free variables. Nodes are only equal if the tag equals. Accessing static or class variables from inside function parameters can be ruled out due to the constraints listed in section \ref{sec:constraints}. $\blacksquare$

Note that this can not be easily proven in a pure formal way, since this would require a theoretical model for the whole Scala language. 

[Todo: Define which operations a optimal algorithm is allowed to make. (Re-Ordering, (Re-)Execution, Deletion)]
\begin{definition}[Optimal change propagation algorithm]
Let $A$ be an optimal change propagation algorithm. That means that $A$ re-evaluates as few nodes as possible during change propagation. 
Let $\alpha(I, I')$ be the count of re-evaluated nodes for a change propagation from an input $I$ to another input $I'$ for the same program. 
\end{definition}

Let $I$ and $I'$ be two inputs for a program. Let $T$ and $T'$ be the traces of the program execution with $I$ and $I'$ as input. 
To show that our trace distance algorithm finds a lower bound for change propagation, we have to show that $\delta(T, T') = \alpha(I, I')$. 

\begin{lemma}
\label{lem:alphaLeq}
$\delta(T, T') \leq \alpha(I, I')$
\end{lemma} 
Proof: Let $Y$ be the set off all nodes of $T$ without a cognate, let $R$ be the set of all nodes of $T'$ without a cognate. 
That means, that for all nodes in $Y$ ther is no corresponding node with an equal tag in trace $T'$ and vice versa. Therefore, we have to at least re-evaluate all nodes within $Y$ and $R$, whereas the removal of a node counts as re-evaluation. Thus, the count of re-evaluated nodes is greater or equal to the trace distance. $\blacksquare$

\begin{lemma}
\label{lem:alphaGeq}
$\delta(T, T') \geq \alpha(I, I')$
\end{lemma} 
Proof: Let $B$ be the set off al nodes in $T$ and $T'$ which have a cognate and are therefore equal. Lets assume that ther is a vertex $v$ in $B$ which is re-executed during change propagation by the optimal algorithm $A$. Due to lemma \ref{lem:equalExec} we know that this re-execution was unnecassary. Thereforew, our assumtion was wrong. We know now that an optimal algorithm does not re-evaluate vertixes which have a cognate, or in other words, it may only re-evaluate vertices which have no cognate. A the count of all vertices without a cognate equals the trace distance, we know that the count of re-evaluated nodes is lower or equal than the trace distance. $\blacksquare$

\begin{theorem}
$\delta(T, T') = \alpha(I, I')$
\end{theorem}
Proof: Follows directly from lemma \ref{lem:alphaLeq} and \ref{lem:alphaGeq}. $\blacksquare$
\section{Automatic optimization of programs}
This section is going to describe how analyzing the Directed Dependency Graph (DDG) can be used for automatic optimization. For accomplishing this task we can utilize the following features of the DDG: 
\begin{itemize}
\item Caller/callee dependencies. 
\item Dependencies of modifiables\footnote{Pointer-like variables which have to be explicitly read and written, and therefore support automatic change propagation}.
\item Dependencies of bound variables which are not modifiables. 
\end{itemize}

Furthermore, using the intrinsic distance algorithm, we can recognize which nodes are deleted, inserted or retained \cite{Acar2005thesis}, which can be used to optimize the program to accomplish faster change propagation. 

The exact contents described in this chapter are still to be determined, based on our findings. Possible approaches include, but may not be limited to: 

\begin{itemize}
\item Function call reordering. 
\item Insertion of explicit memorization calls.
\item Detection of cascading updates, which could be omitted. 
\end{itemize}
\section{Evaluation}
This section is going to demonstrate the usefulness of the described techniques using real-world algorithms, like map, reduce and quicksort. 

Basically, it is shown how it is possible to optimize a classic implementation (without memorization) of each algorithm, so that change propagation time lies within the same complexity class as the theoretical lower bound for updates for this algorithm. 
\section{Conclusion}
This section will conclude and summarize with the findings of this work. 

\subsection{Future work}
The final section briefly outlines problems encountered but not solved during the writing of this thesis, as well as encourages future research on interesting issues of incremental computation.

 


%\IEEEtriggeratref{8}
\bibliographystyle{IEEEtran}
\bibliography{tbd}
\end{document}