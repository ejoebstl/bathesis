\section{Theoretic fundamentals}
\label{sec:ddg_memo}
This section should give a brief summarization of the principles described in the work of U. Acar et al. This  is important, because we later have to adjust the definitions to fit the program model of TBD. \cite{Acar2005thesis}

\subsection{Execution Traces}
This subsection describes the approaches of using Traces and Directed Dependency Graphs (DDGs). \cite{Acar2005thesis}

\subsection{Memorization}
This subsection describes how traces and memorization together are used to accomplish incremental computing. \cite{Acar2005thesis}

\subsection{Stable algorithms}
This subsection describes the concepts of stable algorithms, intrinsic trace distance and their relationship. 
\footnote{Intrinsic trace distance is a central concept for this work and can basically be described as an edit distance between two trees. 
The definition can be found in \cite{Acar2005thesis}, chapter 7 or \cite{acar2004dynamizing}.}

Also, this section should emphasis that the intrinsic trace distance forms a lower bound for the time needed by change propagation during an update. \cite{Acar2005thesis} 

\subsection{Program model for TBD}
This subsection describes a program model for TBD, which will be used for theoretical conclusions in the following sections. The differences and similarities with the programming model in \cite{Acar2005thesis} should be highlighted.